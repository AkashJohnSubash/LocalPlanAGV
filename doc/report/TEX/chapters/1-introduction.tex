\chapter{Introduction}\label{chap:introduction}
In order to realize the transition from Automated Guidance Vehicles to Autonomous Mobile Robots (AMR), the need for robust, embedded control algorithms is evident.
Relying traditionally on \ac{PID} controllers tuned to track deviation from guiding lanes, \ac{AGV}s are often limited to fixed routes throughout their lifetime which are obstacle-free. 
The benefit of \ac{MPC} for trajectory tracking has been shown in several works \cite{wang_research_2023}, \cite{zhang_trajectory_2021}. By incorporating such a feed-forward model scheme for an intuitive formulation of motion control, we shift focus to meeting kinematic and dynamic constraints along the prediction horizon.

\section{Automated Guided Vehicles}
The intra-logistics industry involving storage and movement of materials within warehouses, and factory floors, has seen a significant increase in automation in the last decade. Since the transport and stacking of packages in a production chain involves a great deal of manual labour and automated processes, identification of the scope of automation is vital to reduce redundancy, transportation and storage costs.
\par \ac{AGV}s have been used to this end to navigate within warehouses and factories to transport heavy materials, as well as sensitive equipment in sterile environments like hospitals, and semiconductor fabrication units. They achieve this feat by following marked routes detected by radio wave sensors, cameras, magnets or laser scanners. Radio signal navigation requires wires embedded into factory floors calling for carefully planned routes, which can be mitigated using magnetic tapes \cite{zhou_development_2020}. These tapes are however prone to wear and must also be placed with care for long routes and crossings. Alternatively, laser navigation offers solutions that only require line-of-sight reflective markers, allowing them to be strategically reorganized with lower overhead. To further minimize the external infrastructure for navigation, Monte-Carlo localization techniques have become popular in robotics \cite{pires_natural_2021}.
\par Obstacle avoidance is a capability that differentiates \ac{AGV}s from AMRs and hence, is a key challenge to be addressed while enabling the transition to the latter. AMRs achieve dynamic path planning by combining global route-finding algorithms and local planners for obstacle evasion. 
 While global planners often rely on heuristic-based designs lying outside the scope of this work, a local motion control scheme to follow a reference track capable of performing evasive manoeuvres around obstacles for tricycle-kinematic \ac{AGV}s at ek robotics is proposed here.

\section{Model Predictive Control}\label{sec:setup}
\ac{MPC} is a feedback-based control scheme with a feedforward model, that has been popular in the process control industry since the late 1900s to tackle challenges in chemical plants and oil refineries, involving a feed-forward model. Within the class of optimization-based control strategies, this optimal controller operates under the assumption of the feasibility of real-time optimization. While the process industry accommodated the luxury of several minutes for such computations, robotics often has more stringent deadlines. This strategy also goes under the name of receding horizon control, which aptly describes the activity of the prediction horizon being dynamic. MPC outshines classical controllers under analytically described constrained operating conditions; while minimizing overshoot due to its prediction capability. An overview of MPCs development over the last few decades can be gauged from Raghu et al. \cite{raghu_model_2013}.
\par The feedforward model used in the scheme must be sufficiently accurate to be used on a plant under realistic operating conditions and simultaneously of sufficiently low complexity allowing the numerical solver to converge. Due to the fast dynamics and control input updates in mechanical applications, using a capable solver framework for deployment of real-time NMPC is required. To address this challenge of repeatedly solving an \ac{OCP} online, CasADi \cite{andersson_casadi_2019} and acados \cite{verschueren_acados_2020} were chosen. CasADi is a modelling and automatic differentiation framework developed at KU Leuven, and acados, a numerical framework for optimal control and estimation algorithms, developed at the University of Freiburg. Acados, which as a successor of the ACADO toolkit developed at KU Leuven, provides modules for the integration of differential equations involving an \ac{ODE} or \ac{DAE} system. Relying on the high-performance linear algebra libraries from BLASFEO \cite{frison_blasfeo_2018}, it offers Python and MATLAB programming interfaces to state-of-the-art QP solvers like HPIPM \cite{frison_hpipm_2020}, qpOASES \cite{ferreau_qpoases_2014}, qpDUNES \cite{frasch_new_2014} and others based on real-time iteration schemes. Acados generates C code tailored to embedded architectures for the computational efficiency of optimal control problems, which we capitalize on to implement the control scheme on an \ac{AGV}. We initally solve the \ac{NLP} with Direct Multiple Shooting \cite{bock_multiple_1984} with IPOPT \cite{wachter_implementation_2006} to validate the approach, and subsequently as QP subproblems in acados.
\par Chapter \ref{chap:background} guides the reader through the mathematical theory established for \ac{MPC} in Section \ref{back_opt_control}, and motivates an appropriate approach to system modelling in Section \ref{back_modelling}. It ends with the related work and highlights our contribution in Section \ref{back_related}. Chapter \ref{chap:approach} goes on to describe candidate kinematic models in Section \ref{appr_kinematics}, detail the Optimal Control Problems in Section \ref{appr_fren_traj}, and its associated real world considerations in Section \ref{appr_oper_safe}. Chapter \ref{chap:experiments} illustrates the simulation and warehouse test environment in Section \ref{expr_gazebo} used along with metrics to evaluate the performance of the two approaches in Section \ref{expr_KPI}. Finally, Chapter \ref{chap:conclusion} concludes this work with remarks on future scope. 