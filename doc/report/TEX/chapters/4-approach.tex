\chapter{Approach}\label{chap:approach}
To enable the \ac{AGV}s to navigate in the warehouse, we investigate different vehicle models intended for a real-time N\ac{MPC} deployment.

\section{Analytic model}\label{appr_kinematics}
In this section, we outline various tricycle kinematic models, which suffice for the scenario in which the \ac{AGV} navigates in a warehouse excluding highly dynamic maneuvres, and within the velocity bounds where wheel slippage can be safely ignored.

\subsection{Tricycle kinematics}
\begin{figure}[htbp]
	\begin{center}
        \def\svgwidth{0.7\textwidth}
        \input{../figures/kin_cart2.pdf_tex}
        \caption{Tricycle kinematic model relating the \ac{AGV} and global Cartesian frames.}
        \label{kin_tricycle}
	\end{center}
\end{figure}
\begin{sloppypar}
The Cartesian state vector of the AGV with tricycle kinematics described by $\zeta^{c}$ = [$x$, $y$, $\varphi$]$\mathrm{^{\top}}$ $\in$ $\mathbb{R}$$\mathrm{^{3}}$ represents the position and heading in the global reference frame. It is further augmented by the vector $\zeta^{u}$ = [$v$, $\alpha$]$\mathrm{^{\top}}$ $\in$ $\mathbb{R}$$\mathrm{^{2}}$, to restrain the rate of $\zeta^{u}$, by defining the control input as  $u$ = [$a$, $\omega$]$\mathrm{^{\top}}$ $\in$ $\mathbb{R}$$\mathrm{^{2}}$. The control vector represents the wheel acceleration $a$, and wheel turning rate $\omega$ which are the first-order time derivatives of the wheel speed $v$ and heading $\varphi$ respectively. This vector $u$ defined in the vehicle reference frame with axes $x_{\mathrm{agv}}$ and $y_{\mathrm{agv}}$ is projected to the global reference frame using the rotation matrix $M (\varphi)$ at the centre of gravity (CG). Although the resulting extended state vector $\zeta^{p}$ = [$x$, $y$, $\varphi$, $v$, $\alpha$]$\mathrm{^{\top}}$ $\in$ $\mathbb{R}$$\mathrm{^{5}}$ for pose tracking and control vector $u$ are continuous-time functions, this dependence is omitted further for notational convenience. The nonlinear kinematics for this system, defined purely in the Cartesian coordinate frame, results in the following \ac{ODE}s, which are adapted from \cite{ljubi_path_2023}. The wheelbase parameter $d$, depicted in Figure \ref{kin_tricycle} common to all three models is defined in \hyperref[table1]{Table I}.
\end{sloppypar}
\begin{align}
    \dot{\zeta^{p}}(t) = \begin{bmatrix}
        \dot{x}\\
        \dot{y}\\
        \dot{\varphi}\\
        \dot{v}\\
        \dot{\alpha}\\
    \end{bmatrix} & = 
    \begin{bmatrix}
        v\, \cos(\alpha)\, \cos(\varphi)\\
        v\, \cos(\alpha)\, \sin(\varphi)\\
        \dfrac{v}{d}\, \sin(\alpha)\\
        a\\
        \omega \label{eqZetaC}
    \end{bmatrix}
\end{align}
\begin{sloppypar}
For a more intuitive representation of track progress and lateral displacement $s$ and $n$ respectively, along a predefined reference curve, we switch to the Frenet coordinate frame. Here, we adapt the dynamic model from \cite{kloeser_nmpc_2020} to the AGV's tricycle kinematics, using the state vector $\zeta^{f}$ = [$s$, $n$, $\beta$]$\mathrm{^{\top}}$ $\in$ $\mathbb{R}$$\mathrm{^{3}}$. From the reference curve definition $\upgamma(s)^T$ = [$\upgamma^{P}(s)^{T}$, $\upgamma^{\varphi}(s)^{T}$], $\upgamma^{\mathrm{P}}(s)$  = $[\upgamma^{x}(s), \upgamma^{y}(s)]^T$ is described on the warehouse floor by b-splines with a curvature $\kappa(s)$ =  $\dfrac{d \upgamma^{\varphi}(s)}{d{s}}$. Subsequently, the resulting state vector $\zeta^{d}$ evolution is described by the \ac{ODE}.
\end{sloppypar}
\begin{align}
    \dot{\zeta^{d}}(t) = \begin{bmatrix}
        \dot{s}\\
        \dot{n}\\
        \dot{\beta}\\
        \dot{v}\\
        \dot{\alpha}\\
    \end{bmatrix} &= 
    \begin{bmatrix}
        \dfrac{v\, \cos(\alpha)\, \cos(\beta)}{1 - n\, \kappa(s)}\\
        v\, \cos(\alpha)\, \sin(\beta)\\
        \dfrac{v}{d}\, \sin(\alpha) - \dfrac{\kappa(s)\, v\, \cos(\alpha)\, \cos(\beta)}{1 - n\, \kappa(s)}\\
        a\\
        \omega  \label{eqZetaD}
    \end{bmatrix}.
\end{align}

\par To profit from the advantages of both coordinate frames, an extended state vector, derived in the lifted Frenet formulation of \cite{reiter_frenet-cartesian_2023}, is tailored here to a tricycle kinematic model. Details of the index reduction of the ensuing \ac{DAE} outlined further in the Section \ref{IdxRed}, shows that the previously defined $\zeta^{c}$, $\zeta^{f}$, $\zeta^{u}$ can be used to obtain the dual coordinate frame state

\begin{align}
    \dot{\zeta^{l}}(t) = \begin{bmatrix}
        \dot{x}\\
        \dot{y}\\
        \dot{\varphi}\\
        \dot{s}\\
        \dot{n}\\
        \dot{\beta}\\
        \dot{v}\\
        \dot{\alpha}\\
    \end{bmatrix} &= 
    \begin{bmatrix}
        v\, \cos(\alpha)\, \cos(\varphi)\\
        v\, \cos(\alpha)\, \sin(\varphi)\\
        \dfrac{v}{d}\, \sin(\alpha)\\
        \dfrac{v\, \cos(\alpha)\, \cos(\beta)}{1 - n\, \kappa(s)}\\
        v\, \cos(\alpha)\, \sin(\beta)\\
        \dfrac{v}{d}\, \sin(\alpha) - \dfrac{\kappa(s)\, v\, \cos(\alpha)\, \cos(\beta)}{1 - n\, \kappa(s)}\\
        a\\
        \omega \label{eqZetaL}
    \end{bmatrix}.
\end{align}

The common control vector for the above formulations is
\begin{align}
u(t) &= 
\begin{bmatrix}
    a \\
    \omega \label{eqU}
\end{bmatrix}.
\end{align}

\begin{figure}[h!tbp]
    \begin{center}
        \def\svgwidth{0.75\textwidth}
        \input{../figures/kin_fren2.pdf_tex}
        \caption{Relation between Cartesian and Frenet frames for a curvilinear track $\upgamma(s)$.}
        \label{cart_fren_tf}
    \end{center}
\end{figure}

\begin{table}[htbp]\label{table1}
	\caption{Parameters of the model}
    \small
	\begin{center}
        \begin{tabular}{lccccl}\toprule
		    \textbf{Symbol} & \textbf{Value} & \textbf{Description}\\
            \midrule
            $\kappa$ & [-0.4, 0.4] $\mathrm{m}^{-1}$ & Track curvature \\
            $d$ & 1.03 $\mathrm{m}$ & Wheelbase \\
            $l_{\mathrm{agv}}$ & 2.914 $\mathrm{m}$ & AGV length \\
            $b_{\mathrm{agv}}$ & 1.115 $\mathrm{m}$ & AGV breadth \\
		    \bottomrule
		\end{tabular}
	\end{center}
\end{table}

\subsection{Coordinate transforms}
Utilizing the well-known transforms between the Cartesian and Frenet coordinate frames in Equations (\ref{C2F}) and (\ref{F2C}), the measured \ac{AGV} position is used to compute the closest point on the reference curve with respect to $s^{*}$, defined as

\begin{mini}|l|
    {s}{ \lVert p_{\mathrm{agv}} - \upgamma^{p}(s) \rVert^{2}_{2}}{s^{*} = }{}{{\label{Sopt}}}{}.
\end{mini}
\begin{align}
    \zeta^{f} = \mathcal{F}_{\upgamma}(\zeta^{c}) &=\begin{bmatrix}
        s^{*}\\
        (p_{\mathrm{agv}} - \upgamma^{p}(s^{*}))\, e_{n}\\
        \varphi - \upgamma^{\varphi}(s^{*}) \label{C2F}
    \end{bmatrix} 
\end{align}
\begin{align}
    \zeta^{c} = \mathcal{F}^{-1}_{\upgamma}(\zeta^{f}) &=\begin{bmatrix}
        \upgamma^{x}(s) - n\, \sin( \upgamma^{\varphi}(s))\\
        \upgamma^{y}(s) + n\, \cos( \upgamma^{\varphi}(s))\\
        \upgamma^{\varphi}(s) + \beta \label{F2C}
    \end{bmatrix} 
\end{align}

Where $p_{\mathrm{agv}}$ - $\upgamma^{p}(s^{*})$ gives the difference in position between the \ac{AGV} and the closest point on the predefined test track, and the unit normal vector $e_{n}$ to the curve.
To describe rotations between the \ac{AGV} and the map reference frames in the cartesian coordinates, the 2D rotation matrix is defined as
\begin{align}
    M(\varphi) &=\begin{bmatrix}
        \cos(\varphi) &-\sin(\varphi)\\
        \sin(\varphi) &\cos(\varphi) \label{2D_rot}
    \end{bmatrix}.
\end{align}
\subsection{Index reduction for dual formulation}\label{IdxRed}
In general, only one of the state representations can be described as \ac{ODE}s, and the other as algebraic equations while transforming between coordinate frames. In systems using Cartesian and Frenet states as here, the choice of \ac{ODE} and algebraic states can be Equations (\ref{eqZetaC}) coupled with (\ref{C2F}), or Equations (\ref{eqZetaC}) coupled with (\ref{F2C}) respectively. Algebraic states, however, introduce \ac{DAE}s, which are tricky to handle since they may introduce inconsistent definitions of boundary conditions \cite{campbell_applications_2019}. DAEs are a class of differential equations that are particularly suitable to represent systems with both differential and algebraic relationships. A differential equation of the form in Equation (\ref{diff_eq}) can be identified as a DAE if its partial derivative is rank-deficient, as indicated by Equation (\ref{rank_diff}).
\begin{align}
    F(\zeta, \dot{\zeta}, u) = 0 \label{diff_eq}
\end{align}
\begin{align}
    \dfrac{\partial{F}}{\partial{\dot{\zeta}}} = 0 \label{rank_diff}
\end{align}
High-index \ac{DAE}s are often index-reduced until they can be forward-simulated stably for use with standard numerical integrators \cite{gros_nonlinear_2012}. Following the argument in \cite{reiter_frenet-cartesian_2023} against using the \ac{DAE} with strongly nonlinear kinematics, we reduce the system to one involving purely differential equations as sketched below. Time differentiating the algebraic relation (\ref{F2C}) between the Frenet differential state and its Cartesian projection, while referring to the Figure \ref{cart_fren_tf} leads to
\begin{subequations}
\begin{align}
    \dot{\zeta}^{c} &= \dfrac{d{\mathcal{F}^{-1}_{\upgamma}(\zeta^{f})}}{dt}\\
    &= \dfrac{\partial{\mathcal{F}^{-1}_{\upgamma}(\zeta^{f})}}{\partial{\zeta}^{f}}\,f(\zeta^{f}, u)\\
    &= \begin{bmatrix}
        \cos(\upgamma^{\varphi}(s))\,(1 - n\,\kappa(s)) &-\sin( \upgamma^{\varphi}(s)) &0\\
        \sin(\upgamma^{\varphi}(s))\,(1 - n\,\kappa(s)) &\cos( \upgamma^{\varphi}(s)) &0\\
        \kappa(s) &0 &1
    \end{bmatrix}
    \begin{bmatrix}
        \dot{s}\\
        \dot{n}\\
        \dot{\beta}
    \end{bmatrix}\\
    &= \begin{bmatrix}
        v\, \cos(\alpha)\, (\cos(\upgamma^{\varphi}(s))\, \cos(\beta) - \sin(\upgamma^{\varphi}(s))\, \sin(\beta))\\
        v\, \cos(\alpha)\, (\sin(\upgamma^{\varphi}(s))\, \cos(\beta) + \cos(\upgamma^{\varphi}(s))\, \sin(\beta))\\
        \dfrac{v}{d}\, \sin(\alpha)      
    \end{bmatrix}\\
    &= \begin{bmatrix}
        v\, \cos(\alpha)\, \cos(\varphi)\\
        v\, \cos(\alpha)\, \sin(\varphi)\\
        \dfrac{v}{d}\, \sin(\alpha)
       \end{bmatrix}.
\end{align}
\end{subequations}
By inspecting this result, we observe that it corresponds to the standard kinematic \ac{ODE} in the Cartesian frame ({\ref{eqZetaC}}), allowing the construction of the purely differential joint state (\ref{eqZetaL}) in Cartesian and Frenet frames.

\section{AGV Operational Safety} \label{appr_oper_safe}
\begin{figure}[tb]
    \begin{center}
        \def\svgwidth{0.95\textwidth}
        \input{../figures/control_archi.pdf_tex}
        \caption{Control architecture for the direct elimination MPC formulation.}
        \label{fig_archi}
    \end{center}
\end{figure}

\begin{figure}[t]
    \begin{center}
        \def\svgwidth{1.05\textwidth}
        \input{../figures/breaking_distance.pdf_tex}
        \caption{Braking distance approximation by the front safety field for the chosen reference velocity $v_{\mathrm{ref}}$ \cite{malitzky_markus_mechanical_nodate}.}
        \label{braking}
    \end{center}
\end{figure}

\subsection{Safety field}\label{safety_field}
The application software abstracted in Figure \ref{fig_archi} implements several underlying mechanisms to ensure personnel safety in the shared human-machine workspace. Tactile sensors located at the base of the \ac{AGV} or a virtual safety field could trigger an emergency stop causing it to become inoperable until manually deactivated. For the intended track, we foresee only forward motion, and hence briefly discuss the safe braking distance $\delta_{b}$ required ahead of the \ac{AGV}. From Newton's first law of motion, considering a decelerration $a$ we arrive at  
\begin{align}
    \delta_{b} = \dfrac{-v^{2}_{\mathrm{agv}}}{2\,a_{\mathrm{agv}}}.
\end{align}

Collision avoidance is then guaranteed in the underlying application layer by extending the vehicle footprint by polygonal fields ahead and behind the vehicle. The front field approximates the forward braking distance $\bar{\delta_{b}}$ as a discontinuous function of the wheel speed $v$, by the specified scanner coefficients for the lateral $\delta_{s}$, and angular offsets $\delta_{\theta}$. This polygon's coordinates can then be described as a function of $\zeta^{c}$ as
\begin{align}
    \begin{bmatrix}
        x_{\mathrm{b}}\\
        y_{\mathrm{b}}
    \end{bmatrix} &=
    \begin{bmatrix}
        x\\
        y
    \end{bmatrix} +
    \begin{bmatrix}
        1.2 + \delta_{s} \cos(\delta_{\theta} + \varphi - \dfrac{3\pi}{4})\\
        0.4 + \delta_{s} \sin(\delta_{\theta} + \varphi - \dfrac{3\pi}{4})\label{polygon_coord}
    \end{bmatrix}.
\end{align}
    
The \ac{AGV} footprint enlarged by this polygon is further overapproximated in the \ac{MPC} constraint formulation to ensure uninterrupted driving, which implies that the controller now considers a safe braking distance.

\par The choice of a constant wheel speed for the upcoming \ac{OCP}s simplifies the geometric constraint formulation, by reducing the problem of accounting for this dynamically growing field to the selection of an operating point as shown in Figure \ref{braking}. 

\subsection{Control thresholding and rate limiting}\label{safe_thr}
To restrict the \ac{AGV} maneuverability within safe operating limits, the drive controller module depicted in Figure \ref{fig_archi} limits the requested twist to predefined thresholds. From the below-defined twist vector $\eta$, defined as a function of state, the vector $\zeta^{u}$ is reconstructed. 

\begin{align}
    \eta = g(\zeta^{u}) =\begin{bmatrix}
        w\\
        \dot{\varphi}
    \end{bmatrix} &= 
    \begin{bmatrix}
        v\, \cos(\alpha)\\
        \dfrac{v}{d}\, \sin(\alpha)\label{twist}
    \end{bmatrix}
\end{align}

This vector is subsequently rate-limited, using finite differences. However, it is important to note that the rate limitation is computed with respect to the vector $\tilde{\zeta^{u}}$, applied at the motors, instead of the requested $\zeta^{u}$. This inconsistency in first-order time derivative approximation between \ac{MPC} and the drive controller could diverge leading to loss of optimality in the controller.

\section{AGV, obstacle modelling}\label{appr_footprint}

Although the Frenet coordinate frame allows an intuitive representation of progress and deviation along a curve, modelling obstacles as geometric constraints in it becomes significantly more challenging compared to the Cartesian frame. Convex geometric shapes when projected into the Frenet frame do not preserve this desirable property, leading to loss of safety guarantees in the \ac{SQP} constraint linearization method, arguing to retain obstacle avoidance constraints in the Cartesian space.
\par Since the provided detection framework is limited to circles and line segment estimates from the laser scanner's point cloud data \cite{habermann_obstacle_2010}, other polygons are estimated by circumscribing circles, providing an overapproximation. This limitation can be overlooked for circular and square obstacles, where the overestimation is tolerable. Another practical limitation stemming from the laser scanner placement is the blind fields at the sides of the vehicle. The obstacles are driven out of the \ac{AGV} field of view while it maneuvers around them, warranting the use of an estimator that mitigates this effect.
The Kalman filter provided onboard the obstacle detector package is augmented to this end to provide an estimate of the largest detected obstacle radius which monotonically decreases during the avoidance maneuver. 
\par The simplest and most intuitive Euclidian distance formulation between circular obstacles and an \ac{AGV} circular footprint approximation is unsuitable for this application due to the large spatially infeasible region circumscribing circles cordons off unnecessarily.
The rectangular footprint of the \ac{AGV} better approximated by a bounding ellipse, or multiple covering circles \cite{reiter_frenet-cartesian_2023}, is investigated in this scenario to the \ac{AGV}.


\begin{figure}[htbp]
    \begin{center}
        \def\svgwidth{0.7\textwidth}
        \input{../figures/agv_el2.pdf_tex}
        \caption{Front view of the AGV's enlarged elliptical footprint \cite{malitzky_markus_mechanical_nodate}.}
        \label{fig_el}
    \end{center}
\end{figure}

With $i \in [1, .., n_{ob}]$ circular obstacle approximations and an elliptical footprint (\ref{eqCircObj}), we impose that the \ac{AGV} ellipse as depicted in Figure \ref{fig_el} does not intersect the area covered by these circles for each shooting node $k$. This is captured by enlarging the AGV semi-axes $\lambda_{i}$ = $\Lambda_{\mathrm{agv}}$ + $r_{ob, i}$ and $\rho_{i}$ = $\varrho_{\mathrm{agv}}$ + $r_{ob, i}$ by the obstacle radii $r_{ob, i}$.
\par While the elliptical AGV approximation with circular obstacles is attractive in terms of the number of constraints, these constraints are severely non-convex due to the solution space the state $\varphi$ is allowed by the kinematics. 
\begin{subequations}
\begin{align}
    \begin{bmatrix}
        \Delta x_{i, k}\\
        \Delta y_{i, k}
    \end{bmatrix}^T
    \Sigma(\zeta_{k})
    \begin{bmatrix}
        \Delta x_{i, k}\\
        \Delta y_{i, k}
    \end{bmatrix} - 1 = 0\label{eqCircObj}
\end{align}
\begin{align}\Sigma(\zeta_{k}) =     
    \mathrm{M}(\varphi_{k})^T
    \begin{bmatrix}
        1/\lambda^{2}_{i} & 0\\
        0 & 1/\rho^{2}_{i}
    \end{bmatrix}
    \mathrm{M}(\varphi_{k})
\end{align}
\end{subequations}

\begin{figure}[htbp]
    \begin{center}
        \def\svgwidth{0.65\textwidth}
        \input{../figures/agv_3c2.pdf_tex}
        \caption{Rear view of the AGV's covering circles depicting their lateral offsets \cite{malitzky_markus_mechanical_nodate}.}
        \label{fig_3c}
    \end{center}
\end{figure}

To retain the narrow \ac{AGV} footprint afforded by ellipses and reduce nonlinearity in the constraint formulation, we turn to covering-circles \cite{khorkov_optimization_2021}, \cite{heppes_covering_1997} of equal radii  $r_{c}$ to obtain the minimum number of circles $n_{c}$ = $\lceil \dfrac{l_{\mathrm{agv}}}{b_{\mathrm{agv}}} \rceil$ = 3 with centres below:
\begin{subequations}
\begin{align}
    x_{j} = x + (\delta_{\mathrm{cg}} + j\, \delta_{\mathrm{c}})\, cos(\varphi)\qquad j = -n_{c} + 2,..,n_{c} - 2\\
    y_{j} = y + (\delta_{\mathrm{cg}} + j\, \delta_{\mathrm{c}})\, sin(\varphi)\qquad j = -n_{c} + 2,..,n_{c} -2
\end{align}
\begin{align}
    \begin{bmatrix}
        \Delta x_{i, j, k}\\
        \Delta y_{i, j, k}
    \end{bmatrix}^T
    \Sigma
    \begin{bmatrix}
        \Delta x_{i, j, k}\\
        \Delta y_{i, j, k}
    \end{bmatrix} - 1 = 0 \label{eqElObj}
\end{align}
\begin{align}\Sigma =
\mathrm{M}(\theta_{i})^T
    \begin{bmatrix}
        1/\lambda^{2}_{i} & 0\\
        0 & 1/\rho^{2}_{i}
    \end{bmatrix}
    \mathrm{M}(\theta_{i})
\end{align}    
\end{subequations}

Conversely, representing obstacles by ellipses centered around its position $p_{i, ob}$, with semi-axes $\Lambda_{i, ob}$, $\varrho_{i, ob}$ and orientation $\theta$, for each shooting node $k$, we impose that the AGV covering circles $j \in [-n_{c} + 2, n_{c} - 2,]$, depicted in Figure \ref{fig_3c} do not intersect the area covered by these ellipses (\ref{eqElObj}) \cite{brito_model_2020}. Contrary to above, we enlarge the obstacle semi-axes $\lambda_{i}$ = $\Lambda_{ob, i}$ + $r_{c}$, and $\rho_{i}$ = $\varrho_{i, ob}$ + $r_{c}$ by the covering circle radii $r_{c}$.
\par By aligning their major axis along the track we facilitate progress maximization due to the higher curvature along this axis compared to circles. Although this approach scales the number of obstacle avoidance constraints by $n_{c}$, the \ac{MPC} node does not exceed its budgeted time, and is hence is chosen for the further \ac{OCP} descriptions. 

\begin{table}[htbp]\label{circule_offsets}
    \small
	\begin{center}
        \begin{tabular}{lccccl}\toprule
		    \textbf{Symbol} & \textbf{Value} & \textbf{Description}\\
            \midrule
            $r_{c}$ & 8 $\cdot$ $10^{-1}$ m & Covering circle radius \\
            $\delta_{cg} $ & 2.54 $\cdot$ $10^{-1}$ m & Central circle offset \\
		    \bottomrule
		\end{tabular}
	\end{center}
    \caption{Distance offsets for the covering circles, computed for $\bar{\delta}_{b}$ = 1m. }
\end{table}

\newpage
\section{Frenet trajectory tracking}\label{appr_fren_traj}
With the possibility to choose between a Cartesian primary frame with Frenet algebraic states (\ref{frenAlg}) or a Frenet primary frame with Cartesian algebraic states (\ref{cartAlg}) indicated below to achieve trajectory tracking, the former is identified as a computationally expensive bi-level problem that would require solving the optimization problem (\ref{Sopt}) at each shooting node for $s^{*}(\zeta^{C}_{k})$. This leads to the first formulation in Section (\ref{DEF_OCP}) using the latter, where this optimization must only be computed once for the state feedback.
\begin{align}
    \begin{split}
        \dot{\zeta}^{p} &= f(\zeta^{p}, u)\\
        0 &= {\zeta}^{f} - \mathcal{F}_{\upgamma}(\zeta^{c}) \label{frenAlg}
    \end{split}
\end{align}
\begin{align}
    \begin{split}
    \dot{\zeta}^{d} &= f(\zeta^{d}, u)\\
    0 &= {\zeta}^{c} - \mathcal{F}^{-1}_{\upgamma}(\zeta^{f}) \label{cartAlg}.
    \end{split}
\end{align}
\begin{table}[htbp]\label{table2}
    \small
	\begin{center}
        \begin{tabular}{lccccl}\toprule
		    \textbf{Symbol} & \textbf{Value} & \textbf{Description}\\
            \midrule
            $\tau_{s} $ & 6 $\cdot$ $10^{-2}\,\mathrm{s}$& sampling time \\
            $\tau_{c} $ & 6 $\cdot$ $10^{-2}\,\mathrm{s}$& MPC deadline \\
            $\tau_{m} $ & 6 $\cdot$ $10^{-2}\,\mathrm{s}$& measurement delay \\
            $\tau_{a} $ & 6 $\cdot$ $10^{-2}\,\mathrm{s}$ & actuation delay \\
            $s_{\mathrm{ref}}$ & 1.2 $\cdot$ $10^{1}\,\mathrm{m}$ & reference progress \\
            $v_{\mathrm{ref}}$ & 8 $\cdot$ $10^{-1}\,\mathrm{m\,s^{-1}}$ & reference wheel speed \\
		    \bottomrule
		\end{tabular}
	\end{center}
    \caption{Parameters for NMPC formulation}
\end{table}
\subsection{ Direct elimination Optimal Control Problem }\label{DEF_OCP}
\par The direct elimination OCP with costs in Frenet for a tricycle kinematic model described by (\ref{eqU}) and (\ref{eqZetaL}), and obstacle avoidance constraints in the projected Cartesian state $\hat{\zeta}^{c}$ = $\mathcal{F}^{-1}_{\upgamma}(\zeta^{f})$ are defined as
\begin{subequations}
\begin{align}
	&L\left(\zeta^{d}_{k}, u_k\right) = \lVert \zeta^{d}_{k}- \zeta^{d}_{k, \mathrm{ref}} \rVert^2_Q + \lVert u_k \rVert^2_R\\
    &E\left(\zeta^{d}_{N}\right) = \lVert \zeta^{d}_{N} - \zeta^{d}_{N, \mathrm{ref}} \rVert^{2}_{Q_{N}}.
\end{align}
\end{subequations}
\begin{mini!}|l|
    {u_0,\zeta^{d}_0,..,u_{N-1},\zeta^{d}_N}{ E\left(\zeta^{d}_{N}\right)+ \sum_{k = 0}^{N-1} L\left(\zeta^{d}_k, u_k\right)}{{\label{DEF_cost}}}{}
    \addConstraint{}{\quad \quad \zeta^{d}_0 - \bar{\zeta^{d}_0}}{=0}
    \addConstraint{}{\quad \quad \zeta^{d}_{k+1} - \phi(\zeta^{d}_k, u_k)}{=0,\hspace{0.9em} k=0,\ldots,N-1}
    \addConstraint{\underline{u} }{\le \quad u_k}{\le \overline{u},\hspace{0.8em} k=0,\ldots,N-1}
    \addConstraint{\underline{w} }{\le \quad v_{k}\, \cos(\alpha_k)}{\le \overline{w},\hspace{0.7em} k=0,\ldots,N-1} \label{constr_veh_vel}
    \addConstraint{\underline{\dot{\varphi}} }{\le \quad \frac{v_k}{d}\, \sin(\alpha_k)}{\le \overline{\dot{\varphi}},\hspace{0.8em} k=0,\ldots,N-1} \label{constr_veh_omg}
    \addConstraint{- \infty }{\le \quad n_{k}\, \kappa(s_{k})}{\le 1,\hspace{0.9em} k=0,\ldots,N-1}\label{constr_proj_uniq}
    \addConstraint{1 }{\le \lVert \textGamma_{\mathrm{def}}(\zeta^{d}_{k}) - p_{\mathrm{ob}} \rVert^{2}_{\Sigma(\cdot)}}{\le \infty,\hspace{0.5em} k=0,\ldots,N-1.}\label{constr_obst_def}
    % \addConstraint{}{}{\qquad\hspace{0.25in} j=0,\ldots,n_{c}-1.}\nonumber
\end{mini!}

\par Where ${\zeta^{d}_{\mathrm{trj}}}:= (\zeta^{d}_{0}, .., \zeta^{d}_{N})$, and ${u_{\mathrm{trj}}}:= (u_0, .., u_{N-1})$ denote the state and control trajectories of the discrete-time system respectively, and $\phi(\zeta^{d}_k, u_k)$ the discretized dynamics. The horizon length and the estimated state of the system at the current time instant $k$ are denoted by $N$ and $\zeta_k$ respectively.\\
Trading off between efficiency and safety, we pick suitable limits for the \ac{AGV} speed and its angular velocity, as seen in Equations (\ref{constr_veh_vel}) and (\ref{constr_veh_omg}). These physically motivated constraints impose more conservative bounds than the underlying application safety module discussed in Section \ref{safe_thr}. This is done to ensure that the optimality of \ac{MPC} is preserved, despite the MPC being agnostic to the application safety bounds.
\par The obstacle avoidance formulation for individual objects discussed in Equations (\ref{eqElObj}) and (\ref{eqCircObj}) are imposed on the projected Cartesian state $\tilde{\zeta}^{c}$ as (\ref{constr_obst_def}) which is defined as a nonlinear mapping of the state vector  $\textGamma_{\mathrm{def}}(\zeta^{d})$. Additionally, $\Sigma(\cdot)$ either represents $\Sigma(\zeta_{k})$ or $\Sigma$ depending on the choice of the footprint (\ref{eqCircObj}) or (\ref{eqElObj}) respectively.Finally, while the obstacle avoidance constraints might demand that the \ac{AGV} strays from the centre line, the constraint (\ref{constr_proj_uniq}) still needs to be fulfilled to guarantee the uniqueness of the projection onto the centreline. An indeterminate solution of (\ref{F2C}) could result in the inner high curvature arch while performing an evasive maneuver, if the \ac{AGV} is driven to the centre of the circle with radius $\dfrac{1}{\kappa(s)}$. This effectively permits only a low centreline deviation on high curvature sections.

\subsection{ Lifted Frenet Optimal Control Problem }\label{LF_OCP}
This dual formulation henceforth referred to as the lifted Frenet approach details an OCP obtained by index reduction of the \ac{DAE} in Section \ref{IdxRed}. It allows the trajectory tracking cost function, and the obstacle avoidance constraints to be defined in the Frenet and Cartesian frames respectively as functions of the control $u$ (\ref{eqU}) and purely differential extended state vector $\zeta^{l}$ (\ref{eqZetaL}) of which $\zeta^{d}$ (\ref{eqZetaD}) is a subset:
\begin{subequations}
\begin{align}
	&L\left(\zeta^{d}_{k}, u_k\right) = \lVert \zeta^{d}_{k}- \zeta^{d}_{k, \mathrm{ref}} \rVert^2_Q + \lVert u_k \rVert^2_R\\
    &E\left(\zeta^{d}_{N}\right) = \lVert \zeta^{d}_{N} - \zeta^{d}_{N, \mathrm{ref}} \rVert^{2}_{Q_{N}}
\end{align}
\end{subequations}
\begin{mini!}|l|
    {u_0,\zeta^{d}_0,..,u_{N-1},\zeta^{d}_N}{ E\left(\zeta^{d}_{N}\right)+ \sum_{k = 0}^{N-1} L\left(\zeta^{d}_k, u_k\right)}{}{}
    \addConstraint{}{\quad \quad \zeta^{f}_0 - \mathcal{F}_{\upgamma}(\zeta^{c}_0)}{=0}\label{constr_drift}
    \addConstraint{}{\quad \quad \zeta^{d}_0 - \bar{\zeta^{d}_0}}{=0}
    \addConstraint{}{\quad \quad \zeta^{l}_{k+1} - \phi(\zeta^{l}_k, u_k)}{=0,}{\hspace{0.9em} k=0,\ldots,N-1}
    \addConstraint{\underline{u}}{\le \quad u_k}{\le \overline{u},\hspace{0.8em} k=0,\ldots,N-1}
    \addConstraint{\underline{w}}{\le \quad v_{k}\, \cos(\alpha_k)}{\le \overline{w},\hspace{0.7em} k=0,\ldots,N-1}\label{constr_vel_agv}	
    \addConstraint{\underline{\dot{\varphi}}}{\le \quad \frac{v_k}{d}\, \sin(\alpha_k)}{\le \overline{\dot{\varphi}},\hspace{0.8em} k=0,\ldots,N-1}\label{constr_omega_agv}
    \addConstraint{-\infty}{\le \quad n_{k}\, \kappa(s_{k})}{\le 1,\hspace{1em} k=0,\ldots,N-1}
    \addConstraint{1}{\le \lVert \textGamma_{\mathrm{lf}}(\zeta^{l}_{k}) - p_{\mathrm{ob}} \rVert^{2}_{\Sigma(\cdot)}}{\le \infty,\hspace{0.5em} k=0,\ldots,N-1.}\label{constr_obst_lf}
    % \addConstraint{}{}{\qquad\hspace{0.25in} j=0,\ldots,n_{c}-1.}\nonumber
\end{mini!}
With a fairly similar structure to the direct elimination formulation from Section \ref{DEF_OCP}, an additional equality (\ref{constr_drift}) is sufficient to ensure that the states (\ref{eqZetaC}) stemming from the index reduction do not drift. A more rigorous treatment of stabilization techniques for index reduction can be found in \cite{campbell_applications_2019}.\\
Finally, the obstacle avoidance inequality (\ref{constr_obst_lf}) allows for better constraint linearization and handling, since the position $\textGamma_{\mathrm{lf}}(\zeta^{l})$ is a linear mapping of the differentiable state $\zeta^{l}$ as opposed to the nonlinear mapping $\textGamma_{\mathrm{def}}(\zeta^{d})$ in Equation (\ref{constr_obst_def}).

\par Despite the above two \ac{OCP}s highlighting only individual obstacle consideration for notational simplicity, it's worthwhile to indicate that the parameterization framework in acados was utilized to regard multiple objects simultaneously, whose pose, velocity and radius were tracked by the \ac{KF} in the detector package.
\par A noteworthy observation from the above two \ac{OCP} formulations was the vehicle acceleration during obstacle avoidance maneuvres since the QP attempts to minimize the deviation from the centre-line, while tracking a constant wheel speed. This implicitly demands that the vehicle speed constrained in Equation (\ref{constr_veh_vel}) increases. 
\subsection{State estimation and communication delays}
\hfill \break
\begin{figure}[t]
    \begin{center}
        \def\svgwidth{0.75\textwidth}
        \input{../figures/state_observer.pdf_tex}
        \caption{Observer and delay modelling in the actuation and measurement path for the lifted Frenet formulation.}
        \label{zeta_obs}
    \end{center}
\end{figure}
To realistically handle latency we identify the dominating factors as delays due to expensive nonlinear \ac{MPC} computation and communication networks.
The inherent hassle of measuring one-way latency is overcome by considering the round trip delay observed experimentally due to the Controller Area Network (CAN) bus. Under the assumption that it equally influences the actuation and measurement paths as $\tau_{a}$, and $\tau_{m}$ respectively, the controller is designed to compensate these delays to retain optimality. 

\par The state estimated by the \ac{SLAM} and encoder modules are defined in the \ac{AGV} and Cartesian reference frames respectively, as seen in Figure \ref{fig_archi}. However, since the \ac{OCP} additionally requires an estimate of Frenet state $\zeta^{f}$ to minimize the objective function, we introduce an observer to compute the forward transform defined in Equation (\ref{C2F}) as shown in the Figure \ref{zeta_obs}.
\par Since this feedback mechanism also implicitly constrains the state drift discussed in Section \ref{LF_OCP}, the equality constraint (\ref{constr_drift}) is made redundant and can be omitted from the lifted Frenet \ac{OCP}.