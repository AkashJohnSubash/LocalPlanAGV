% \chapter{Related Work}\label{chap:relatedwork}

% Trajectory tracking for automotive applications with well-defined lanes has been vastly researched in the recent past in the interest of autonomous driving. In this Chapter, we summarize some of the available literature that influenced this work, which involves modelling the dynamics, the track and obstacles appropriately to achieve real-time deployment of the control scheme on the test \ac{AGV}. Apart from introducing the tricycle Kinematic model in the Cartesian frame, \cite{ljubi_path_2023} introduces path-planning algorithms based on heuristic approaches. The fundamentals for lane-bounded time optimal control established using a single track model in the Frenet frame with \ac{SQP} algorithms demonstrated in the work of Verschueren et al. \cite{verschueren_towards_2014}, is improved upon in  \cite{kloeser_nmpc_2020} by Klöser et al. Here, the lateral acceleration is modelled explicitly with a progress maximization objective reformulation relevant to racing, and road boundary deflection. Improved parametrization of reference tracks to avoid singularities in the Frenet-Cartesian transforms is subsequently presented in detail by Reiter et al. in \cite{reiter_parameterization_2021}.

% \par An optimization-based planning scheme with obstacle evasion known as the \ac{TEB} by Rosmann et al. \cite{rosmann_efficient_2013} made popular with an open source implementation in \ac{ROS}, demonstrated the feasibility of compute-intensive planners at EK Robotics in Mobile Robots for a hospital environment. This laid the groundwork for investigating the deployment of more rigorous collision avoidance techniques. The \ac{TEB} is further extended to maintaining multiple candidate topologies with an approximated kinematic model for car-like vehicles in \cite{rosmann_integrated_2017}. A potential field method, modelling objective terms representing repulsive fields promoting evasive manoeuvres is introduced by Jiang et al. in \cite{jiang_obstacle_2016} for autonomous road vehicles. Subsequently, the burden of tuning such methods compared to geometric obstacle constraints is discussed by Xing et al. in \cite{xing_vehicle_2022}. Strict collision avoidance assurance, promising safety in human-machine interaction unlike in the previous approaches, can be better realized when formulated with hard constraints such as ellipses by Brito et al. in \cite{brito_model_2020}, multiple circles by Galiev et al. in \cite{galiev_optimization_2019}, and hyperplanes by Brossette et al. in \cite{brossette_collision_2017}. The single-track bicycle model tested on miniature race cars at the University of Freiburg is further extended in simulation exploring multiple \ac{ODE} formulations in the Cartesian and Frenet frames by Reiter et al. in \cite{reiter_frenet-cartesian_2023} motivated by retaining convex constraint specification for circular and elliptical objects while using the implicit curvilinear states of the Frenet frame. Employing the native DAE formulation stemming from this joint state representation by Xing et al. in \cite{xing_vehicle_2022} uses linear MPC for trajectory tracking. For a more realistic environment representation, an approach introducing non-circular convex polygonal obstacles for unmanned aerial vehicles by Zhang et al. in \cite{zhang_enlarged_2023} elaborates on geometric approximations for quadratic programming instead of Mixed Integer Programming (MIP).

% \par Delving into the practical limitation of retaining optimality of \ac{MPC} in systems with latency inherent in communication networks, Carlos et al. argue \cite{carlos_efficient_2020} the placement of delay nodes for perfect delay compensation. Kartal et al.\cite{kartal_distributed_2020} investigate the latency distribution in multi-agent systems.

% \par Motivated by this plethora of existing research, we aim to design a real-time feasible NMPC scheme for an \ac{AGV} at EK Robotics, validated in simulation and on the test vehicle using acados. We provide some insights into real-time behaviour aboard the test setup, with quantifiable performance metrics.