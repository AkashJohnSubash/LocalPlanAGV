\chapter{Conclusion}\label{chap:conclusion}

The optimal controllers designed in this work have been experimentally validated for real-time performance, proving their feasibility for tricycle kinematic \ac{AGV}s in a warehouse environment. By demonstrating the real-time feasibility of acados for NMPC, the groundwork for the motion control scheme has been laid, paving the way for its extension to production vehicles with various kinematic models.
\par This work highlights a methodical approach to implement two optimal control schemes on an industrial vehicle.
A significant effort is spent investigating and developing the appropriate model for the task, motivated by physical motion constraints and safety mechanisms.    
Starting with IPOPT to validate the control scheme, we then proceed to use acados to achieve the closed loop control in a time and computation-bound environment.
In order to realistically gauge the control performance, we progress to a higher fidelity model in Gazebo, by incorporating the \ac{MPC} node into ek robotics' application software. This involves appropriately utilizing the state feedback from wheel encoders, and localization data from the onboard \ac{SLAM} stack to design a suitable state observer and delay compensation scheme.
\par With a feasible trajectory tracking scheme verified in the real warehouse, we then proceed to conservative constraint formulations for the large \ac{AGV} footprint, considering approximations of real obstacles in the test environment. This not only involves selecting an open-source obstacle detector and abstracting laser scan data to handle multiple objects, but also identifying and using an appropriate estimator to provide obstacle predictions for occluded objects. Finally, we compare the obstacle constraint formulations and controller performance in Gazebo and on the real vehicle \ac{AGV}.
\par The two controllers exhibit similar performance for the AGV covering circles, but the lifted Frenet indicates better performance for the more nonlinear formulation, in the elliptical footprint, highlighting its improved constraint handling.
\par Obtaining more rigorous guarantees on obstacle avoidance, however, calls for a more rugged obstacle detection mechanism that can provide a full field of view among others. Another foreseen improvement is that; the simplified obstacle constraints introduced in the Cartesian frame could further be extended to more generic convex polygons, better representing the working environment of such vehicles. 